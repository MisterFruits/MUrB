%The $n$-body problem is a classic one from the Newtonian mechanics: it consists in resolving gravitation equations.
%However, this problem can be generalized (from the algorithmic point of view) and we can often find it in numerical simulation: this is why it is a good real case study.

%At the problem beginning (the time $t$), for each body $i$, its position $\vec{q_t}(i)$, its mass $m_i$ and its velocity $\vec{v_t}(i)$ are known.
%The applied force between two bodies $i$ and $j$, at the $t$ time, is defined by:
%\begin{equation}
%\label{eq:force}
%	\vec{F_t}(i,j) = G.\frac{m_i.m_j}{(D_{i,j})^2}.\vec{u},
%\end{equation}
%with $G$ the gravitational constant ($G = 6,67384\times10^{-11} m^3.kg^{-1}.s^{-2}$), $D_{i,j}$ the distance between $i$ body and $j$ body and $\vec{u}$ the unit vector pointing from $i$ to $j$.
%The resulting force for a given $i$ body, at the $t$ time, is defined by:
%\begin{equation}
%\label{eq:sumForces}
%	\vec{F_t}(i) = \sum_{j \ne i}^{n} \vec{F_t}(i,j),
%\end{equation}
%with $n$ the space bodies number.
%The acceleration characteristic for a given $i$ body, at the $t$ time, is defined by:
%\begin{equation}
%\label{eq:acceleration}
%	\vec{a_t}(i) = \frac{\vec{F_t}(i)}{m_i}.
%\end{equation}
%The $i$ body velocity characteristic at the $t + dt$ time depends on the velocity and the acceleration at the $t$ time:
%\begin{equation}
%\label{eq:vitesse}
%	\vec{v_{t+dt}}(i) = \vec{v_{t}}(i) + \vec{a_t}(i).dt.
%\end{equation}
%At the end, the $i$ body position $q$ at the $t + dt$ time depends on the position, the speed and the acceleration at the $t$ time:
%\begin{equation}
%\label{eq:position}
%	\vec{q_{t+dt}}(i) = \vec{q_{t}}(i) + \vec{v_{t}}(i).dt + \frac{\vec{a_t}(i).dt^2}{2}.
%\end{equation}
%Thanks to Eq.~\ref{eq:force},\ref{eq:sumForces}, \ref{eq:acceleration}, \ref{eq:vitesse} and \ref{eq:position}, it is now possible to compute the new position and the new velocity for all $i$ bodies at the $t + dt$ time.The $n$-body problem is a classic one from the Newtonian mechanics: it consists in resolving gravitation equations.
%However, this problem can be generalized (from the algorithmic point of view) and we can often find it in numerical simulation: this is why it is a good real case study.

%At the problem beginning (the time $t$), for each body $i$, its position $\vec{q_t}(i)$, its mass $m_i$ and its velocity $\vec{v_t}(i)$ are known.
%The applied force between two bodies $i$ and $j$, at the $t$ time, is defined by:
%\begin{equation}
%\label{eq:force}
%	\vec{F_t}(i,j) = G.\frac{m_i.m_j}{(D_{i,j})^2}.\vec{u},
%\end{equation}
%with $G$ the gravitational constant ($G = 6,67384\times10^{-11} m^3.kg^{-1}.s^{-2}$), $D_{i,j}$ the distance between $i$ body and $j$ body and $\vec{u}$ the unit vector pointing from $i$ to $j$.
%The resulting force for a given $i$ body, at the $t$ time, is defined by:
%\begin{equation}
%\label{eq:sumForces}
%	\vec{F_t}(i) = \sum_{j \ne i}^{n} \vec{F_t}(i,j),
%\end{equation}
%with $n$ the space bodies number.
%The acceleration characteristic for a given $i$ body, at the $t$ time, is defined by:
%\begin{equation}
%\label{eq:acceleration}
%	\vec{a_t}(i) = \frac{\vec{F_t}(i)}{m_i}.
%\end{equation}
%The $i$ body velocity characteristic at the $t + dt$ time depends on the velocity and the acceleration at the $t$ time:
%\begin{equation}
%\label{eq:vitesse}
%	\vec{v_{t+dt}}(i) = \vec{v_{t}}(i) + \vec{a_t}(i).dt.
%\end{equation}
%At the end, the $i$ body position $q$ at the $t + dt$ time depends on the position, the speed and the acceleration at the $t$ time:
%\begin{equation}
%\label{eq:position}
%	\vec{q_{t+dt}}(i) = \vec{q_{t}}(i) + \vec{v_{t}}(i).dt + \frac{\vec{a_t}(i).dt^2}{2}.
%\end{equation}
%Thanks to Eq.~\ref{eq:force},\ref{eq:sumForces}, \ref{eq:acceleration}, \ref{eq:vitesse} and \ref{eq:position}, it is now possible to compute the new position and the new velocity for all $i$ bodies at the $t + dt$ time.


The $n$-body problem is a classic one from the Newtonian mechanics: it consists in resolving gravitation equations.
However, this problem can be generalized (from the algorithmic point of view) and we can often find it in numerical simulation: this is why it is a good real case study.

At the problem beginning (the time $t$), for each body $i$, its position $q_{i}(t)$, its mass $m_i$ and its velocity $\vec{v_i}(t)$ are known.
The applied force between two bodies $i$ and $j$, at the $t$ time, is defined by:
\begin{equation}
\label{eq:force}
	\vec{f_{ij}}(t) = G.\frac{m_i.m_j}{||\vec{r_{ij}}||^2}.\frac{\vec{r_{ij}}}{||\vec{r_{ij}}||},
\end{equation}
with $G$ the gravitational constant ($G = 6,67384\times10^{-11} m^3.kg^{-1}.s^{-2}$), $\vec{r_{ij}} = q_j(t) - q_i(t)$ is the vector from body $i$ to body $j$.
%the distance between $i$ body and $j$ body and $\vec{u}$ the unit vector pointing from $i$ to $j$.
The resulting force (alias the total force) for a given $i$ body, at the $t$ time, is defined by:
\begin{equation}
\label{eq:sumForces}
	\vec{F_i}(t) = \sum_{j \ne i}^{n} \vec{f_{ij}}(t) = G.m_i.\sum_{j \ne i}^{n}\frac{m_j.\vec{r_{ij}}}{||\vec{r_{ij}}||^3},
\end{equation}
with $n$ the number of bodies in space.
For the time integration, the acceleration is needed. For a given $i$ body, at the $t$ time, the acceleration characteristic is defined by:
\begin{equation}
\label{eq:acceleration}
	\vec{a_i}(t) = \frac{\vec{F_i}(t)}{m_i} = G.\sum_{j \ne i}^{n}\frac{m_j.\vec{r_{ij}}}{||\vec{r_{ij}}||^3}.
\end{equation}

\subsection{An approximation of the total force}

The total force $\vec{F_i}$ is given by Eq.~\ref{eq:sumForces}:
\begin{equation*}
	\vec{F_i}(t) = G.m_i.\sum_{j \ne i}^{n}\frac{m_j.\vec{r_{ij}}}{||\vec{r_{ij}}||^3}.
\end{equation*}
As bodies approach each other, the force between them grows without bound, which is an undesirable situation for numerical integration. 
In astrophysical simulations, collisions between bodies are generally precluded; this is reasonable if the bodies represent galaxies that may pass right through each other. 
Therefore, a softening factor $\epsilon^2 > 0$ is added, and the denominator is rewritten as follows:
\begin{equation}
\label{eq:sumForcesSoft}
	\vec{F_i}(t) \approx G.m_i.\sum_{j = 1}^{n}\frac{m_j.\vec{r_{ij}}}{(||\vec{r_{ij}}||^2 + \epsilon^2)^\frac{3}{2}}.
\end{equation}

Note the condition $j\ne i$ is no longer needed in the sum, because $\vec{f_{ii}} = 0$ when $\epsilon ^2 > 0$.
The softening factor models the interaction between two Plummer point masses: masses that behave as if they were spherical galaxies.
In effect, the softening factor limits the magnitude of the force between the bodies, which is desirable for numerical integration of the system state.

As before, we need to compute the acceleration in order to perform the integration over the time:
\begin{equation}
\label{eq:accelerationSoft}
	\vec{a_i}(t) = \frac{\vec{F_i}(t)}{m_i} \approx G.\sum_{j = 1}^{n}\frac{m_j.\vec{r_{ij}}}{(||\vec{r_{ij}}||^2 + \epsilon^2)^\frac{3}{2}}.
\end{equation}